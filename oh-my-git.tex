%!TEX program = xelatex
\documentclass{beamer}
\usepackage[backend=biber,style=numeric-comp,sorting=none]{biblatex}
\addbibresource{biblio.bib}

\usecolortheme[light,accent=orange]{solarized}

\usepackage{subcaption}
\usepackage{gitdags}

\usepackage{blindtext}
\usepackage{xltxtra}
\usepackage{wrapfig}

\usepackage{listings}
\usepackage{multicol}
\usepackage{verbatim}

\usepackage{graphicx,changepage}

\newcommand{\adjustimg}{% Horizontal adjustment of image
  \checkoddpage%
  \ifoddpage\hspace*{\dimexpr\evensidemargin-\oddsidemargin}\else\hspace*{-\dimexpr\evensidemargin-\oddsidemargin}\fi%
}
\newcommand{\centerimg}[2][width=\textwidth]{% Center an image
  \makebox[\textwidth]{\adjustimg\includegraphics[#1]{#2}}%
}

\defaultfontfeatures{Ligatures=TeX}

\title{Oh my GIT!}
\subtitle{Git ate my work and other stories}
\author{Vojtěch Vladyka}
\date{\today}

% \setbeamercolor
%
% solarizedBase03
% solarizedBase02
% solarizedBase01
% solarizedBase00
% solarizedBase0
% solarizedBase1
% solarizedBase2
% solarizedBase3
% solarizedYellow
% solarizedOrange
% solarizedRed
% solarizedMagenta
% solarizedViolet
% solarizedBlue
% solarizedCyan
% solarizedGreen

\begin{document}
    \frame{\titlepage}
    \begin{frame}
       \frametitle{Table of contents}
       \begin{enumerate}
           \item What is git
           \\   \textcolor{solarizedRebase01}{\footnotesize\hspace{1em} ...and why you should care}	
           \item Setup git
           \\   \textcolor{solarizedRebase01}{\footnotesize\hspace{1em} Clients, differences, common issues}	
           \item Git essentials
           \\   \textcolor{solarizedRebase01}{\footnotesize\hspace{1em} Git controls crash course}	
           \item Scenarios
           \\   \textcolor{solarizedRebase01}{\footnotesize\hspace{1em} We are going to break things there. A lot.}	
           \item Q \& A
           \\   \textcolor{solarizedRebase01}{\footnotesize\hspace{1em} Umm, why was I here again?}
       \end{enumerate}
    \end{frame}

    \begin{frame}
        \frametitle{What is git?}
        \begin{itemize}
            \item version control system made by Linus Torvalds (mostly) at 2005 for Linux Kernel development
            \item always capture state of working tree
            \item focused on distributed development
            \item supports rapid branching \& merging
            \item it is pronounced GIT with G like GIF ~\footfullcite{linus-speech}
        \end{itemize}
    \end{frame}
    \begin{frame}
        \frametitle{What is git?}
        \framesubtitle{Differences between Git and SVN}
        \begin{itemize}
            \item Git is distributed by design
            \item Works with whole sourcetree (unlike SVN which works with individual files and their revisions)
            \item Strong support for non-linear development $\rightarrow$ rapid branching \& merging
            \item Cryptographic authentication of history - every commit has SHA-1 hash of everything leading to that point
        \end{itemize}
    \end{frame}
    
    \begin{frame}
        \frametitle{Setup git}
        \framesubtitle{Git clients}
        \begin{enumerate}
                % icons for platforms
                % add short desc to each
            \item Git (all platforms)
            \item Git Extensions (Windows, Mac)
            \item Tortoise Git (Windows)
            \item Sourcetree (Windows)
            \item Fork (Windows, Mac)
            \item Gitkraken (all platforms)
            \item ...many others
        \end{enumerate}
    \end{frame}
    \begin{frame}
        \frametitle{Setup git}
        \framesubtitle{Git Extensions (Git client)}
        \noindent\centerimg[width=\paperwidth]{gitex.png} 
        Download it here $\rightarrow$ \url{https://gitextensions.github.io}
    \end{frame}

    \begin{comment}
    \begin{frame}
        \begin{tikzpicture}
            % Commit DAG
            \gitDAG[grow right sep = 2em]{
                A -- B -- {
                  C,
                  D -- E,
                }
            };
            % Tag reference
            \gittag
                [v0p1]      % node name
                {v0.1}      % node text
                {above=of A}% node placement
                {A}         % target
            % Remote branch
            \gitremotebranch
                [originmaster]
                {origin/master}
                {above=of C}
                {C}
            % Branch
            \gitbranch
                {master}
                {above=of E}
                {E}
            % HEAD reference
            \gitHEAD
                {above=of master}
                {master}
        \end{tikzpicture}
    \end{frame}
    \end{comment}
    \begin{frame}[fragile]
        \frametitle{Git essentials}
        \framesubtitle{Short and incomplete command overview}
        \begin{multicols}{2}
        \begin{lstlisting}[gobble=12]
            git init                
                clone

                fetch
                merge
                push
                pull

                branch
                checkout
                merge
                branch -d


            git log                
                log --follow [file]
                reflog

                add
                commit

                reset
                reset --hard
                stash

                rebase
                cherrypick
        \end{lstlisting}
        \end{multicols}
    \end{frame}
    \begin{frame}
        \frametitle{Git essentials}
        \framesubtitle{Workflow overview}
        \noindent\centerimg[width=\textwidth]{workflow.png}
    \end{frame}
    \begin{frame}[fragile]
        \frametitle{Git essentials}
        \framesubtitle{Init repo}
        \begin{lstlisting}[gobble=12]
            git init
        \end{lstlisting}
        $\rightarrow$ initialize new repository in current directory with all its contents
        \begin{lstlisting}[gobble=12]
            git clone https://somerepo.com/repo.git
        \end{lstlisting}
        $\rightarrow$ downloads repository with all current branches \& commits
    \end{frame}
    \begin{frame}[fragile]
        \frametitle{Git essentials}
        \framesubtitle{Fetching changes and returning them back}
        \begin{lstlisting}[gobble=12]
            git pull
        \end{lstlisting}
        $\rightarrow$ Download commits from server and apply them to your working tree
        \begin{lstlisting}[gobble=12]
            git push
        \end{lstlisting}
        $\rightarrow$ Upload your commits to server
        \begin{lstlisting}[gobble=12]
            git fetch --all
        \end{lstlisting}
        $\rightarrow$ Download commits from server (from all branches eventually)
        \begin{lstlisting}[gobble=12]
            git commit [files] -m "Commit message"
        \end{lstlisting}
        $\rightarrow$ Create a commit with commit message and specified files\newline
        $\rightarrow$ You can use -a to commit all changes instead list of files
        \begin{lstlisting}[gobble=12]
            git add [file]
        \end{lstlisting}
        $\rightarrow$ Move file to staging area - add it to commit list
    \end{frame}
    \begin{frame}[fragile]
        \frametitle{Git essentials}
        \framesubtitle{Undoing changes}
        \begin{lstlisting}[gobble=12]
            git reset
        \end{lstlisting}
        $\rightarrow$ Soft reset - remove files from staging area and move them back to working area. Nothing is actually reverted, this resets it only for git.
        \begin{lstlisting}[gobble=12]
            git reset --hard
        \end{lstlisting}
        $\rightarrow$ Hard reset - this do all what soft reset but also REMOVES your uncommitted changes.
        \newline\newline
        $\rightarrow$ You can do this only locally. Once you push your changes, you cannot get it back. (not exactly true but you will break a lot of thing for all colleagues). Only way how to propperly revert things it using git revert.
    \end{frame}

    \begin{frame}[fragile]
        \frametitle{Git essentials}
        \framesubtitle{Branching}
        \begin{lstlisting}[gobble=12]
            git checkout -b feature1
            (git branch feature1 ; git checkout feature1)
        \end{lstlisting}
        \resizebox{!}{0.25\paperheight}{%
        \begin{tikzpicture}
            % Commit DAG
            \gitDAG[grow right sep = 2em]{
                A -- B -- C -- {
                    D,
                    E -- F
                }
            };
            % Branch
            \gitbranch
                {master}
                {above=of D}
                {D}
            % Branch
            \gitbranch
                {feature1}
                {above=of F}
                {F}
            % HEAD reference
            \gitHEAD
                {above=of feature1}
                {feature1}
        \end{tikzpicture}%
        }
        \begin{lstlisting}[gobble=12]
            git checkout master
            git merge feature1
        \end{lstlisting}
        \resizebox{!}{0.25\paperheight}{%
        \begin{tikzpicture}
            % Commit DAG
            \gitDAG[grow right sep = 2em]{
                A -- B -- C -- {
                    D,
                    E -- F
                } -- G
            };
            % Branch
            \gitbranch
                {master}
                {below=of G}
                {G}
            % Branch
            \gitbranch
                {feature1}
                {below=of F}
                {F}
            % HEAD reference
            \gitHEAD
                {below=of master}
                {master}
        \end{tikzpicture}%
        }

    \end{frame}
    \begin{frame}[fragile]
        \frametitle{Git essentials}
        \framesubtitle{Logging and searching lost stuff}
        \begin{lstlisting}[gobble=12]
            git log --graph
        \end{lstlisting}
        $\rightarrow$ Shows log for whole repository tree
        \begin{lstlisting}[gobble=12]
            git log --follow [file]
        \end{lstlisting}
        $\rightarrow$ Shows log for selected file even across renames
        \begin{lstlisting}[gobble=12]
            git reflog
        \end{lstlisting}
        $\rightarrow$ Shows log of ALL changes in repository. Stashes, detached branches, all...
    \end{frame}
    \begin{frame}[fragile]
        \frametitle{Git essentials}
        \framesubtitle{Stashing - let me just put this here for later (and never use it again)}
        \begin{lstlisting}[gobble=12]
            git stash
            git stash push
        \end{lstlisting}
        $\rightarrow$ Stashes current changes to new stash stack
        \begin{lstlisting}[gobble=12]
            git stash pop
        \end{lstlisting}
        $\rightarrow$ Pops newest stash and apply it to current working copy
        \begin{lstlisting}[gobble=12]
            git stash list
        \end{lstlisting}
        $\rightarrow$ Show all stashes in list
        \begin{lstlisting}[gobble=12]
            git stash apply stash@{2}
        \end{lstlisting}
        $\rightarrow$ Get stash \#2 and apply it to current working copy
        \begin{lstlisting}[gobble=12]
            git checkout stash@{2} -- somefile
        \end{lstlisting}
        $\rightarrow$ Get stash \#2 and apply it to specified files in current working copy
    \end{frame}
    \begin{frame}[fragile]
        \frametitle{Git essentials}
        \framesubtitle{Rebase}
        \begin{lstlisting}[gobble=12]
            git rebase target_branch source_branch
        \end{lstlisting}
        \resizebox{!}{!}{%
        \begin{tikzpicture}
            % Commit DAG
            \gitDAG[grow right sep = 2em]{
                A -- B --{
                    C -- D ,
                    E -- F -- G
                }
            };
            % Branch
            \gitbranch
                {master}
                {above=of D}
                {D}
            % Branch
            \gitbranch
                {feature1}
                {above=of G}
                {G}
            % HEAD reference
            \gitHEAD
                {above=of feature1}
                {feature1}
        \end{tikzpicture}%
        }
        \begin{center}$\Downarrow$\end{center}
        \resizebox{!}{!}{%
        \begin{tikzpicture}
            % Commit DAG
            \gitDAG[grow right sep = 2em]{
                A -- B -- C -- D -- {
                    E' -- F' -- G'
                }
            };
            % Branch
            \gitbranch
                {master}
                {above=of D}
                {D}
            % Branch
            \gitbranch
                {feature1}
                {above=of G'}
                {G'}
            % HEAD reference
            \gitHEAD
                {above=of feature1}
                {feature1}
        \end{tikzpicture}%
        }
        \end{frame}
        \begin{frame}[fragile]
            \frametitle{Git essentials}
            \framesubtitle{Rebase}
            \begin{lstlisting}[gobble=12]
                git cherry-pick <commit hash>
            \end{lstlisting}
            \resizebox{!}{!}{%
            \begin{tikzpicture}
                % Commit DAG
                \gitDAG[grow right sep = 2em]{
                    A -- B --{
                        C -- D ,
                        E -- F -- { G }
                    } 
                };
                % HEAD reference
                \gitHEAD
                    {above=of D}
                    {D}
            \end{tikzpicture}%
            }
            \begin{center}$\Downarrow$\end{center}
                \resizebox{!}{!}{%
            \begin{tikzpicture}
                % Commit DAG
                \gitDAG[grow right sep = 2em]{
                    A -- B --{
                        C -- D -- { F' },
                        E -- F -- { G }
                    } 
                };
                % HEAD reference
                \gitHEAD
                    {above=of F'}
                    {F'}
            \end{tikzpicture}%
            }
            \end{frame}
    \begin{frame}
        \frametitle{Q \& A}
    \end{frame}
    \section{Děkujeme za pozornost}
\end{document}